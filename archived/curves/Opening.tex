%%%%%%%%%%%%  Generated using docx2latex.com  %%%%%%%%%%%%%%

%%%%%%%%%%%%  v2.0.0-beta  %%%%%%%%%%%%%%

\documentclass[12pt]{article}
\usepackage{xeCJK}
\usepackage{amsmath}
\usepackage{latexsym}
\usepackage{amsfonts}
\usepackage[normalem]{ulem}
\usepackage{soul}
\usepackage{array}
\usepackage{amssymb}
\usepackage{extarrows}
\usepackage{graphicx}
\usepackage{longtable}
\usepackage[backend=biber,
style=numeric,
sorting=none,
isbn=false,
doi=false,
url=false,
]{biblatex}\addbibresource{bibliography.bib}

\usepackage{subfig}
\usepackage{wrapfig}
\usepackage{wasysym}
\usepackage{enumitem}
\usepackage{adjustbox}
\usepackage{ragged2e}
\usepackage[svgnames,table]{xcolor}
\usepackage{tikz}
\usepackage{longtable}
\usepackage{changepage}
\usepackage{setspace}
\usepackage{hhline}
\usepackage{multicol}
\usepackage{tabto}
\usepackage{float}
\usepackage{multirow}
\usepackage{makecell}
\usepackage{fancyhdr}
\usepackage[toc,page]{appendix}
\usepackage[hidelinks]{hyperref}
\usetikzlibrary{shapes.symbols,shapes.geometric,shadows,arrows.meta}
\tikzset{>={Latex[width=1.5mm,length=2mm]}}
\usepackage{flowchart}\usepackage[paperheight=11.69in,paperwidth=8.27in]{geometry}
\usepackage{CJKutf8}
\TabPositions{0.29in,0.58in,0.87in,1.16in,1.45in,1.74in,2.03in,2.32in,2.61in,2.9in,3.19in,3.48in,3.77in,4.06in,4.35in,4.64in,4.93in,5.22in,5.51in,}

\urlstyle{same}


 %%%%%%%%%%%%  Set Depths for Sections  %%%%%%%%%%%%%%

% 1) Section
% 1.1) SubSection
% 1.1.1) SubSubSection
% 1.1.1.1) Paragraph
% 1.1.1.1.1) Subparagraph


\setcounter{tocdepth}{5}
\setcounter{secnumdepth}{5}


 %%%%%%%%%%%%  Set Depths for Nested Lists created by \begin{enumerate}  %%%%%%%%%%%%%%


\setlistdepth{9}
\renewlist{enumerate}{enumerate}{9}
		\setlist[enumerate,1]{label=\arabic*)}
		\setlist[enumerate,2]{label=\alph*)}
		\setlist[enumerate,3]{label=(\roman*)}
		\setlist[enumerate,4]{label=(\arabic*)}
		\setlist[enumerate,5]{label=(\Alph*)}
		\setlist[enumerate,6]{label=(\Roman*)}
		\setlist[enumerate,7]{label=\arabic*}
		\setlist[enumerate,8]{label=\alph*}
		\setlist[enumerate,9]{label=\roman*}

\renewlist{itemize}{itemize}{9}
		\setlist[itemize]{label=$\cdot$}
		\setlist[itemize,1]{label=\textbullet}
		\setlist[itemize,2]{label=$\circ$}
		\setlist[itemize,3]{label=$\ast$}
		\setlist[itemize,4]{label=$\dagger$}
		\setlist[itemize,5]{label=$\triangleright$}
		\setlist[itemize,6]{label=$\bigstar$}
		\setlist[itemize,7]{label=$\blacklozenge$}
		\setlist[itemize,8]{label=$\prime$}



 %%%%%%%%%%%%  Header here  %%%%%%%%%%%%%%





%%%%%%%%%%%%%%%%%%%% Document code starts here %%%%%%%%%%%%%%%%%%%%



\begin{document}
\begin{Center}
{\fontsize{16pt}{19.2pt}\selectfont \textbf{毕业论文(设计)开题报告}\par}
\end{Center}\par



%%%%%%%%%%%%%%%%%%%% Table No: 1 starts here %%%%%%%%%%%%%%%%%%%%


\begin{longtable}{p{0.58in}p{0.98in}p{0.29in}p{0.39in}p{0.39in}p{0.01in}p{0.18in}p{0.59in}p{0.29in}p{0.1in}p{0.35in}}
\hline
%row no:1
\multicolumn{1}{|p{0.58in}}{\Centering 论文题目} &
\multicolumn{10}{|p{5.37in}|}{直线与椭圆、双曲线的位置关系及性质探讨} \\
\hhline{-----------}
%row no:2
\multicolumn{1}{|p{0.58in}}{\Centering 学生姓名} &
\multicolumn{1}{|p{0.98in}}{} &
\multicolumn{1}{|p{0.29in}}{\Centering 系别} &
\multicolumn{1}{|p{0.39in}}{\Centering 数学系} &
\multicolumn{1}{|p{0.39in}}{\Centering 专业} &
\multicolumn{3}{|p{1.18in}}{\Centering 数学与应用数学} &
\multicolumn{1}{|p{0.29in}}{\Centering 班级} &
\multicolumn{2}{|p{0.64in}|}{} \\
\hhline{-----------}
%row no:3
\multicolumn{1}{|p{0.58in}}{\cellcolor[HTML]{FFFFFF}\Centering 学生学号} &
\multicolumn{1}{|p{0.98in}}{} &
\multicolumn{2}{|p{0.88in}}{\Centering 指导教师姓名} &
\multicolumn{2}{|p{0.6in}}{} &
\multicolumn{1}{|p{0.18in}}{\Centering 职称} &
\multicolumn{1}{|p{0.59in}}{} &
\multicolumn{2}{|p{0.59in}}{\Centering 所属单位} &
\multicolumn{1}{|p{0.35in}|}{\Centering 数学系} \\
\hhline{-----------}
%row no:4
\multicolumn{11}{|p{6.15in}|}{一、选题的目的和意义: \par 

对整系数和有理系数多项式方程,即Diophantine方程,的解的研究可追溯到古典希腊时代。而现代Diophantine几何是通过代数数论和代数几何方法对Diophantine方程进行研究的学科。
本文涉及三类二元Diophantine方程,即线性方程(直线)
$$aX+bY+c=0,\quad a,b,c\in\mathbb{F}\quad a\neq0\lor b\neq0;$$
二次方程(抛物线、椭圆、双曲线等)
$$aX^2+bXY+cY^2+dX+eY+f=0\quad a,\dots,f\in\mathbb{F}\quad a\neq0\lor b\neq0\lor c\neq0;$$
和一类特殊的三次方程——其定义的代数曲线具有一基点并亏格为一,这被称作椭圆曲线。本文中,我们约定$F=\mathbb{Z},\mathbb{Q},\mathbb{R},\mathbb{C},\mathbb{F}_q,\text{or }\mathbb{Z}_l$。

以椭圆曲线为例,我们可以进行
\colorbox{SkyBlue}{\href {https://arxiv.org/search/advanced?advanced=\&terms-0-term=elliptic+curve\&terms-0-operator=AND\&terms-0-field=all\&terms-1-term=math.AG\&terms-1-operator=AND\&terms-1-field=all\&classification-mathematics=y\&classification-physics_archives=all\&classification-include_cross_list=include\&date-filter_by=past_12\&date-year=\&date-from_date=\&date-to_date=\&date-date_type=submitted_date_first\&abstracts=show\&size=200\&order=-announced_date_first} {arXiv搜索}}
来证明椭圆曲线作为代数几何的研究方向的今年热度

} \\
\hhline{~~~~~~~~~~~}
%row no:5
\multicolumn{11}{|p{6.15in}|}{\ \  } \\
\hhline{-----------}
%row no:6
\multicolumn{11}{|p{6.15in}|}{二、本课题的研究现状: \par } \\
\hhline{~~~~~~~~~~~}
%row no:7
\multicolumn{11}{|p{6.15in}|}{\ \ \  } \\
\hhline{-----------}
%row no:8
\multicolumn{11}{|p{6.15in}|}{三、主要内容和预期目标: \par 



考虑二次整系数方程
$$aX^2+$$






} \\
\hhline{~~~~~~~~~~~}
%row no:9
\multicolumn{11}{|p{6.15in}|}{\ \ \  } \\
\hhline{-----------}
%row no:10
\multicolumn{11}{|p{6.15in}|}{四、拟采用的研究方法和主要措施: \par } \\
\hhline{~~~~~~~~~~~}
%row no:11
\multicolumn{11}{|p{6.15in}|}{\ \   } \\
\hhline{-----------}
%row no:12
\multicolumn{11}{|p{6.15in}|}{五、主要参考文献: \par \ \ \  } \\
\hhline{~~~~~~~~~~~}
%row no:13
\multicolumn{11}{|p{6.15in}|}{\ \ \  } \\
\hhline{-----------}
%row no:14
\multicolumn{11}{|p{6.15in}|}{六、指导教师意见: \par 指导教师签名:\ \ \ \ \  \ \ \ \ \ \ \ \  \  \ 年\ \ \ 月   日 \par } \\
\hhline{-----------}
%row no:15
\multicolumn{11}{|p{6.15in}|}{七、指导小组意见: \par 组长签名:\ \  \ \  \ \ \  \ \ \ \ \ \ \ \ 年\ \ \ 月   日 \par } \\
\hhline{-----------}

\end{longtable}


%%%%%%%%%%%%%%%%%%%% Table No: 1 ends here %%%%%%%%%%%%%%%%%%%%

注:此表由学生填写。开题报告会结束后,由指导教师和小组签署意见。论文答辩前,学生将此表交指导教师。此表按要求装订在论文文本内。\par

\printbibliography
\end{document} 