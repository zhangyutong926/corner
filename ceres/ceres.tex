\documentclass[12pt]{report}
\usepackage{xeCJK}
\usepackage{amssymb}
\usepackage{amsthm}
\usepackage{mathtools}
\usepackage{amsmath}
\usepackage{amsfonts}
\usepackage{enumitem}
\usepackage{tikz-cd}
\usepackage{array}
\usepackage{makecell}
\usepackage{tabularx}
\usepackage{geometry}
\usepackage{multicol}
\usepackage{titling}
\geometry{a4paper,scale=0.9}

\setlength{\parindent}{0cm}
\setlength{\parskip}{1em}
\newcommand*{\qedfill}{\hfill\ensuremath{\blacksquare}}
\newcommand{\hookdoubleheadrightarrow}{%
  \hookrightarrow\mathrel{\mspace{-15mu}}\rightarrow
}
\newcommand{\hookdoubleheadleftarrow}{%
  \leftarrow\mathrel{\mspace{-15mu}}\hookleftarrow
}
\renewcommand{\setminus}{\mathbin{\backslash}}

\title{Sayako}
\begin{document}
\maketitle

\chapter{Ouroboros}
\color{red}
这是一个单人模组,主题是数学,时长通常在半小时内\footnote{\color{red}当然KP也可以把$N$设定的很大让模组变长...}。
在一周目请不要告知PL游戏标题(Ouroboros)。
KP在游戏开始前应决定一个常数$N$,即迭代的最大次数。
定义数列$\{a_{n,m}\}_{n=0,m=1}^{N,3}=-\frac{1}{N}(n+\frac{m}{4})+1$。
KP可以设定PC的同期,考虑到有些PL会倾向于使用暴力解决问题,对同期们的战斗属性的设定可能是必要的\footnote{\color{red}毕竟不是所有人都想阿虚那样有耐心,如果$N$很大的话。不过这是个无用选项,世界线总会收束的。}。
KP可以决定游戏的难度:游戏中数学部分可以给PL提供真正的数学题目,按照游戏迭代过程由现代变为近代,或按照下述方式进行【科学-数学】检定。
关于数学的部分可以向作者Sayako索要适合不同PL的题目,例如高等微积分、分析学、近世代数、微分几何、代数几何等。

\color{green}
迭代开始($m=1$):

\color{black}
这是一个普通的早晨,你像往常一样尝试赖床几分钟。
你发现你无法再次睡着,于是便百无聊赖地开始回想昨天的梦境。
你能依稀记得那是一个计算机实验室,你明确地记起一段对白:
“三味大学「D-Module」\footnote{\color{blue}一个没什么卵用的信息:这个名字来源于一个同名的数学概念,D-module,可自行google。}的读数是$a_{n,m}$!”
\color{blue}
过一个【灵感】或【幸运】检定,成功可以回想起昨晚睡前接了一个陌生人的电话\footnote{\color{red}致敬一下Steins;Gate...},\color{red}sc$1$d$3$\color{blue}。

\color{red}
其实所谓的D-Module是一个类似世界线检测器的东西,不同的是它的0--1之间的值代表了人类对数学和自然科学的理解。
正如脚注所述,那通电话的确是一个时间机器的接受器,它带给了PC这个奇怪的梦境,但在$n=0,m=1$时它没有改变世界线。
这通电话是一个——如果用SCP基金会的术语的话——「自我实现的悖论」(或「自指悖论」,取决于翻译的不同):每次发送D-Mail\footnote{\color{red}使用Steins;Gate的术语只是为了方便KP理解,请不要对PL使用这些术语。}的原因是之前接收到的D-Mail,所以$n=0,m=1$时,第一次的D-Mail在某种意义上还未被发送。
若PC在第二天向任意一人透露了关于梦境的信息,则自我实现的悖论循环完整,D-Module的指数将会单调递减,PC所在世界的技术奇点无法到达。

\color{black}
你想起了三味大学正是你要申请的学校,而今天是开放式题目的提交日期。
你发现了桌上没做完的题目。
\color{red}
请KP提供第$n$次迭代的题目或过检定$\min\{\,\text{【科学-数学】}\mkern-6mu-n\operatorname{d}10,1\}$。
若解出题目或通过检定则游戏继续,若失败且$n=0$则进入HE,否则进入BE。

\color{black}
你因为解出了开放式题目而成功被三味大学研究生院数学系录取。
\color{red}
请想象力丰富的KP在这里填一些好玩的剧情和RP...理科生真的写不出东西来哇...
\color{black}
你现在可以在三味大学开始展开调查。大部分学生会表示他们不知道D-Module相关的信息。
\color{blue}
可通过【幸运】检定直接从某学生口中获得关于NPC-A与D-Module项目有关的信息。

\color{green}
前往导师办公室/Homeroom的剧情:

\color{black}
导师在第一天没有office hour,不会在办公室。
你发现导师办公室有一把你没有见过的密码锁。
\color{blue}
除非通过【开锁】困难检定(或PL真的说出了523264这个密码),否则在第一天这个门无法被打开。
若门被打开,则可通过一次【侦察】检定提供给PL导师的文件的信息:
\begin{enumerate}
\item 为了探知达到理论中存在的技术奇点的可能性,三味大学数学系向三味大学法人提出D-Module计划。
\item D-Module结合世界已有的量子计算技术,分析当今人类的数学和基础科学水平,得出D-Module指数。
\end{enumerate}

\color{green}
关于NPC-A的剧情:

\color{red}
NPC-A是个名副其实的疯狂科学家\footnote{\color{red}我又开始Steins;Gate了...}。
\color{black}
你发现与NPC-A的交谈异常地困难,因为ta总沉浸在自己的想法中。
\color{blue}
通过【母语】检定后,可以从NPC-A口中得知关于ta正在进行的TemporaLink项目,它将一段内存共享给自从它创建以来的全部时间点,即任何对它的修改将会改变过去,使得这段内存在它加电时开始就具有这段被改变的信息。NPC-A声称ta正在对ta的“时间机器”进行测试:在加电启动前,时间机器已被编程为显示全部共享内存中的内容——这代表着ta可以向过去发送信息。
\color{red}
得到这些信息的PC,sc1d6。

\color{green}
第一天结束后的梦境:

\color{black}
在第一天结束后,你会再次获得相同的梦境。
除了$a_{0,2}$这个数字以外,
\color{blue}
可通过分别通过【灵感】、【灵感】和【侦察】发现NPC-A、NPC-C和导师NPC-B。
若PC已经见过NPC-A,那么第一个【灵感】则无需检定。
若能够通过【意志】困难鉴定,则梦境可为PC所用,可根据探索提供以下线索:
\begin{enumerate}
\item 看到了导师使用实验室的密码523264\footnote{\color{blue}一个没什么卵用的信息:这个数字是19维同伦球面的h-配边类的数量,可以考虑通过【灵感】和【科学-数学】检定提供给好奇的PL...},通过【灵感】或PL联想可以在第二天打开导师办公室的门;
\item NPC-C与NPC-A是同组成员,\color{red}NPC-C是可攻略角色(嘛其实不是真的romantic的那种攻略,只是增加信任度之类的吧),且其对PC的好感度会为其提供新的结局,可向PL暗示这一点\color{blue};
\item D-Module在望月数学中心地下B101室。
\end{enumerate}
\color{red}
若PC做出意外(大意)的举动,KP可选择不提供一部分线索,并强制其醒来,线索中断。

\color{green}
第二天($m=2$)前往D-Module的房间:

\color{red}
随机出现NPC-A、NPC-B、NPC-C。
\color{blue}
需通过极难【说服】检定才可获得关于D-Module有关的信息(导师办公室的文件),NPC会要求玩家严格保密。若【说服】普通失败,则被退学,脱落。
若PC选择说出关于梦境的信息,则NPC-B立刻意识到这是ta的技术——通过手机的声音向人类大脑传递潜意识信息\color{red}(若PL还没有得到这条信息的话)\color{black}。
\color{blue}
PL可自行联想或通过【灵感】得知关于时间机器和自指悖论的全貌。
\color{black}
NPC-B说服了NPC-A向你提供使用TemporaLink完成自指悖论的机会。
\color{red}
若PC选择使用,则将PC的经历发送给过去的PC,进入下一次迭代,$n=n+1$。
若PC选择不使用,游戏继续。

\color{green}
一些关于迭代的细节:

\color{red}
众所周知,基础科学的水平直接影响了应用科学的水平。
虽然在这个游戏中,我们设定D-Module和TemporaLink总是存在于每次迭代,但对于没能获得导师的文件的PC,KP可以提供一些周边信息来暗示PL关于D-Module指数的意义,例如使用CoC的时代技术对应(变回1920什么的...)。

\color{green}
第二天($m=2$)去找NPC-C:

\color{black}
你发现对比NPC-A的疯狂,NPC-B的固执,NPC-C是一个平易近人的角色。
NPC-C对你表示ta对NPC-A和NPC-B的行事风格的不满,并认为应对D-Module的研究成果进行公开。
\color{red}
然后进入喜闻乐见的刷好感度环节,方式KP决定,例如:约会,做题,进行【科学-数学】检定等。

\color{green}
第三天($m=3$):

\color{red}
若$m=N$,且PC决定前往数学中心B101,则进入TE。
否则若PC还未透露关于梦境的信息,且NPC-C的好感度足够,进入SemiHE,否则进入BE。

\color{green}
HE:

\color{black}
由于未能提交正确的开放式题目答案,你未能进入三味大学就读。
你也无从查找那个神秘的电话的来源。
但你仍然保持了对数学的兴趣,并在几年后成功申请到了另一所大学的数学系。
你目睹了现代数学发展的黄金时期和技术奇点的到达。
在遥远的未来,你和你同时代的人已经远去的时代,在AI和人类的共同努力下,这个世界的数学和科学仍然在蓬勃发展。

\color{purple}【科学-数学】$+1\operatorname{d}6$

\color{green}
BE:

\color{black}
在这个世界,人类的数学和基础科学水平指数为$a_{n,3}$,未能为奇点提供有效的基础。在NPC-A的坚持下,D-Module被销毁,{当前年份+10}被历史成为“现代数学的最后一年”,在那之后学习现代数学至可以进行研究所需的时间超出了人类的生命周期。
见证了这一切的你束手无策,在孤独中结束了自己的生命。

\color{green}
SemiHE:

\color{black}
在这个世界,人类的数学和基础科学水平指数为$a_{n,3}$,未能为奇点提供有效的基础,在
值得庆幸的是,由于你和NPC-C的努力,D-Module相关的研究被发表于世,直到人类最后的岁月里,他们还仍然记得你和NPC-C的名字。

\color{purple}【科学-数学】$+1\operatorname{d}3$

\color{green}
TE:

\color{black}
你来到望月数学中心B101室,只不过D-Module已经消失了。
房间空荡荡的,但有一个人(NPC-D)好像正等待着你的到来。
NPC-D告诉你ta的到来是为了帮助D-Module指数归零的世界的数学家。
NPC-D向你提出来前往ta的世界的邀请,并说明ta的世界已经经历了D-Module的自指悖论并成功脱困。
你跟随NPC-D来到ta的世界,并在那里完成了自己的博士学位,过上了平凡数学家的生活。

\color{purple}【科学-数学】$+2\operatorname{d}6$

\end{document}
