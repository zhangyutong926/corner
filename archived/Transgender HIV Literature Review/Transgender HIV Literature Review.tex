\documentclass[5p]{elsarticle}
\usepackage{xeCJK}
\usepackage[hidelinks]{hyperref}
\journal{Beijing LGBT Center Library of Transgenderism}
\bibliographystyle{model5-names}\biboptions{authoryear}
\usepackage{etoolbox}
% \patchcmd{<cmd>}{<search>}{<replace>}{<success>}{<failure>}
\patchcmd{\abstract}{Abstract}{摘要}{}{}
\patchcmd{\keyword}{Keywords}{关键词}{}{}
\setlength{\parskip}{0.1cm}
\setlength{\parindent}{2em}
\setCJKmainfont[BoldFont=FandolSong-Bold.otf,ItalicFont=FandolKai-Regular.otf]{FandolSong-Regular.otf}
\setCJKsansfont[BoldFont=FandolHei-Bold.otf]{FandolHei-Regular.otf}
\setCJKmonofont{FandolFang-Regular.otf}

\begin{document}
\begin{frontmatter}
\title{跨性别和HIV文献综述}
\author[1]{Jian Min}
\ead{jianmis@uci.edu}
\author[2]{Zhang Yutong}
\ead{knight.of.lambda.calculus@gmail.com}
\fntext[fn1]{作者姓名按字母顺序排序}
\end{frontmatter}

在过去的20年中,``跨性别者"一词已经成为描述拥有多种性别体验的个体的总称,他们中的大部分人拥有与他们出生时的指派性别不一致的性别认同。在很大程度上由于对全球HIV疫情的广泛关注和响应,跨性别的主观性在被于全球公共健康的讨论中边缘化的同时被合理化。具体地说,如 \cite{poteat2015}所述,大部分跨性别女性在HIV相关的调查中被分类为男男性行为者(MSM),这经常导致对跨性别群体中的HIV风险的独特性的忽视。

事实上,\cite{poteat2015}认为跨性别人群中的HIV风险是多维的,复杂的,并且常常是多个元素的交互产生的结果------从生物学和医学因素到社会和结构性因素。例如,出生时的指派性别为男性的跨性别女性在与男性伴侣进行无保护的肛交时更容易接触HIV。而对于跨性别男性,睾酮的使用可能导致使得在阴道性交中HIV接触几率增加的阴道萎缩。而对于社会和结构性因素,作为使跨性别者决定进入性工作产业以维持生计的部分原因,结构性的羞辱和歧视,对HIV的意识不强,以及拥有多个性工作产业中的性伴侣一同将跨性别个体置于了一个对于HIV风险的高度弱势地位。

如果将所有这些因素一起考虑,\cite{baral2013}描述的在跨性别社群中的``世界范围内的问题"便不会令人惊讶。具体来说,全世界的跨性别女性,不论其国家的经济发展情况,都有更高的接触HIV的几率和更高的HIV感染流行性。换句话说,在发展中国家和发达国家中的跨性别女性都有接触HIV的高度风险。为了获得对这个问题的更加深度的了解,我们在线面的段落中回顾了相关研究并描述的它们的意义。我们首先描述早期在西方国家进行的研究,然后介绍关于中国跨性别人群中HIV风险的有限现存文献。

在 \cite{modan1992}中,研究者进行了在一个由216名女性和跨性别性工作者组成的未经选择的群体的HIV流行性的研究。样本被问及年龄、生理性别、婚姻状态、后代、执业时长、性行为特质、和药物滥用历史。血液检查在当场进行。所有128位没有承认药物滥用的女性被检测为隐形;52位承认静脉内药物滥用的女性中2位 (3.8\%) 被检测为阴性。作为对比,36位跨性别女性中11\%的样本,包括32位非药物滥用这的3位,被检测为阴性。这个结果支持HIV的阴道传播比肛门传播更加低效的主张。

在 \cite{handzel1990}中,研究者认为认为临床健康无症状的以城市定居的同性恋男性中的高概率免疫系统异常与无保护的接受性直肠性交有关,这些人中的41\%一致地呈现出与接受性肛交无关的T细胞缺陷,并且这些缺陷与HIV抗体的状态独立。研究者比较了一个由14个异装者和男性接受性肛交同性性工作者组成的同质人群和另一个由HIV阴性的非完全被动的拥有多个性伴侣的男性同性恋组成的人群的T淋巴细胞档案和EBV及CMV的平均几何滴度。尽管44\%的控制组呈现了相对于正常异性恋男性的T细胞或它们的子群体的减少,这些性工作者的T淋巴细胞值均处于正常范围内。在这些性工作者中,HIV血清抗体仅在一人中被检测到,CMV血清抗体在13人中的12人检测到,EBV在全体中检测到。尽管存在CMV (86.6) 和EBV (25.8) 的高平均几何滴度,频繁的接受性肛交本身并不足以导致这个群体中的免疫功能受损。 

在 \cite{chew1997}中,研究者展示了277位跨性别的社会特性、性行为、和HIV抗体流行性。研究者使用了标准问卷采访了在新加坡国立大学医院申请性别重置手术的154名跨性别男性和123名跨性别女性。血液样本在预约时和手术前被提取,他们使用了商用酶免疫测定进行HIV抗体测试。研究结果表明,男性和女性跨性别者的平均年龄是26.3和28.3。这些跨性别者主要是新加坡华人并且至少有高中学历。0位跨性别女性和14位 (9\%) 跨性别男性从事性工作行业。显著地多的跨性别女性 (48\%) 比跨性别男性 (10.4\%) 更加性生活不活跃。在研究中,跨性别性行为和实践也被提及。相比于跨性别女性,显著地多的跨性别男性是性活跃的、有更早的性接触、并更加经常地进行肛门性交、且拥有7位或更多性伴侣。然而,没有任何跨性别者是HIV抗体阳性的。

尽管前述研究中没有观察到跨性别个体中的高HIV感染率,\cite{bockting1998}和 \cite{clements1999}提供了关于HIV接触可能的因素以及实施预防的新视角。在 \cite{bockting1998}中,研究者承认尽管临床经验和初步研究显示一些跨性别人群处于HIV的显著风险中,目前这个被污名化的群体的HIV预防是被忽略的。在指向跨性别群体的HIV预防教育的发展过程中,由被选择的跨性别者组成的专门小组评估了他们的HIV风险和预防需求。在这份研究中,研究者收集了以下四个方面的数据:(1) HIV/AIDS对跨性别者的影响;(2) 影响因子;(3) 需要的信息和服务;以及 (4) 样本的招募策略。结果表明HIV/AIDS与已有的针对跨性别者身份认同的污名化相结合,影响出柜过程中跨性别者对性的实验,并且可能影响他们获取性别重置的过程。被本研究确定的针对跨性别的风险因子包括:性别认同冲突、羞耻性和孤立、隐私性、对认同的寻求、强迫性行为、性工作、以及在注射激素时分享注射器。社群介入,研究者强调同伴教育和对跨性别者认同的肯定作为成功的干预的不可分的组成部分,研究者认为对卫生专业人士的关于跨性别认同和性取向的教育以及位跨性别者设置的HIV/AIDS支持团体是急需的。

\cite{clements1999}定性地描述了San Francisco地区跨性别者的HIV风险水平以及他们对HIV预防和卫生服务的使用。在研究中,100位跨性别女性和跨性别男性组成了十一个专门小组。研究者对这些小组进行了抄录和审查,他们的评论被以从数据中自然产生的范畴分类,非重复的评论被编号和总结。研究的结论是,HIV高风险行为,例如未经保护的性行为、商业性工作、和注射药物的使用是常见的。缺乏自尊、经济原因、以及药物滥用被作为进行更加安全的行为的常见阻碍被提及。很多个体因为其他事项以及服务提供者对跨性别特性的不敏感而拒绝使用公共预防和卫生服务。针对如何改善这个状态,参与者的建议包括雇佣跨性别者来开发和实施相关计划并进行针对已有的卫生服务提供者的关于跨性别敏感性和护理标准的培训。

对近10年来发表的研究的回顾表明跨性别群体中的HIV风险已经在学者中获得了更多关注。其结果是学术社区中存在越来越多的共识,他们认为在跨性别群体中应对HIV风险的干预是急需的(\cite{baral2013}; \cite{desantis2009}; \cite{herbst2008}; \cite{nuttbrock2009})。这些学者中的大多数关注跨性别女性的HIV风险,他们认为,她们在顺性别和异性恋的社会中的弱势地位导致了HIV易感行为(例如高风险性行为、性工作等)。举例来说,她们中的多数缺乏精神卫生支持、医疗保险、以及经济稳定性,这有可能增加她们从事高度易于HIV接触的行为。因此,在对跨性别群体的HIV风险评估中,背景因素是不可被忽略的------尤其是对于跨性别女性。确实,为了将跨性别女性的性行为置于环境中并更好地理解她们的挣扎,\cite{melendez2007}进行了与20名前往纽约一家社区诊所的跨性别女性的深度交谈。他们的定性数据揭示了他们研究中的跨性别女性经常寻求顺性别男性的爱与亲密性,这肯定了她们的性别和女性特质。她们与男性伴侣的性行为在一方面为她们提供了爱、接纳性、以及对她们的认同的合理化,却又在另一方面导致了不安全的性行为。很多跨性别女性会遵从她们伴侣的请求进行未经保护的性交来获得一种亲密性和真实性的感受,这导致了她们的HIV风险的增加。因此,在跨性别社区进行工作的学者和卫生专业人士需要比数字更加远见才能真正地理解导致了跨性别HIV弱势的社会文化背景和因素。

尽管世界意识到了跨性别社区的HIV风险,只有极少的研究在中国展开,或许是因为中国跨性别个体的高度隐形性。根据 \cite{cai2016},在他们与沈阳220名跨性别女性的匿名采访中,26.8\%的参与者报告她们有过与男性客户的无套肛门性交,23.2\%的参与者报告她们有过与非客户的男性的无套肛门性交。与男性客户进行无保护肛门性交的原因包括女性化医疗干预(例如整形手术)和感受到的跨性别认同对无套性交的印象。这些因素看起来与 \cite{melendez2007}的关于纽约跨性别女性的研究------尤其是跨性别女性的主观性和女性特质让她们与她们的顺性别男性客户讨论避孕套使用变得困难,这些男性客户推测拥有对渴望他们的爱和感情的跨性别女性更多权力------相似。此外,\cite{cai2016}也提及了几乎没有HIV预防服务提供给中国的跨性别女性,她们中的大部分在中国被视为MSM。这经常导致人们忽略跨性别人群的预防需求。因此,不仅更多关于跨性别人群的研究是必要的,对这个群体的社会接受性和可视性也是急需的,为了将她们从MSM范畴进行区分,从而使得更加敏感和合适的关怀和干涉成为可能。

\section*{引用}
\bibliography{bibfile}
\end{document}
