\documentclass[10pt]{article}
\usepackage{xeCJK}
\usepackage{amssymb}
\usepackage{amsthm}
\usepackage{mathtools}
\usepackage{amsmath}
\usepackage{amsfonts}
\usepackage{enumitem}
\usepackage{tikz-cd}
\usepackage{array}
\usepackage{makecell}
\usepackage{tabularx}
\usepackage[citestyle=authoryear,bibstyle=authortitle,sorting=ynt,backend=bibtex]{biblatex}
\usepackage{geometry}
\usepackage{multicol}
\usepackage{titling}
\addbibresource{template}
\geometry{a4paper,scale=0.9}
\newcounter{counter}
\newcommand{\counter}{\refstepcounter{counter}{\thecounter} }

\setlength{\parindent}{0cm}
\setlength{\parskip}{1em}
\newcommand*{\qedfill}{\hfill\ensuremath{\blacksquare}}

\title{Differential Form and Characteristic Classes by Duan Haibao}
\author{Compiler: Sayako Hoshimiya}
\begin{document}

\maketitle
\renewcommand{\setminus}{\mathbin{\backslash}}

\section{Vector Bundle and Basic Notions}
Vector bundles are a class of fundamental geometric objects that is used to construct spaces and manifolds in topology. It is also the geometric carrier of various kinds of characteristic class theories. In this chapter the notions related to them are reviewed and standards are settled.

\subsection{The Basic Notions of Vector Bundles}
Let $F^n$ be an $n$-dimensional topological vector space. Fix a topological space $B$ and a positive integer $n$, consider the product space $E=B\times F^n$ and the projection to the first factor $p:E\to B$, which is called the \textbf{$n$-dimensional trivial vector bundle} with base space $B$ and denoted as $\varepsilon_B^n$.

Generally, a continuous map $p:E\to B$ determines a partition of $E$, i.e. $$E=\bigsqcup_{b\in B}E_b\quad\text{ where }\quad E_b=p^{-1}(b),b\in B.$$

\textbf{Definition 1 (Vector Bundle).} A triple consisting of $E,B$, and a map $p:E\to B$ is called an \textbf{$n$-dimensional $F$-vector bundle} on $B$ if \begin{enumerate}\item For any $b\in B$, $E_b$ is an $n$-dimensional $F$-vector space;\item There exists an open cover $\{B_a\}_{a\in A}$ of $B$ and a family of homeomorphisms $\{\varphi_a:p^{-1}(B_a)\to B_a\times F^n\}_{a\in A}$ such that the restrictions $$\varphi_a\big|_{E_b}:E_b\to\{b\}\times F^n$$ are linear maps for all $b\in B_a$.\end{enumerate} The set $\{(B_a,\varphi_a)\}_{a\in A}$ is called a \textbf{trivialization} of the vector bundle.

There are common terms regarding a vector bundle $\xi:E\overset{p}{\to}B$:\begin{itemize}\item $\dim\xi=\dim_FE_b$ is called the \textbf{dimension} of $xi$;\item $B$ is called the \textbf{base space} of $\xi$;\item $E$ is called the \textbf{total space} of $\xi$; and\item $E_b$ is called the fiber on the point $b\in B$.\end{itemize}

The continuous map \begin{align*}\sigma:B&\to E\\b&\mapsto 0\in E_b\end{align*} is called the \textbf{zero section} of the bundle $\xi:E\overset{p}{\to}B$. This defines a subspace $$E^0=E\setminus\operatorname{Im}\sigma=\bigsqcup_{b\in B}(E_b\setminus\{0\}).$$

\colorbox{red!30}{\textbf{Proposition S\counter.}} The fibers $E_b$ of a vector bundle are topological vector spaces.
\begin{proof}
For $b\in B_a$ and $e,e'\in E_b$, we have \begin{align*}e+e'&=(\varphi_a^{-1}\circ\varphi_a)(e+e')\\&=(\varphi_a^{-1}\circ\varphi_a\big|_{E_b})(e+e')\\&=\varphi_a^{-1}(\varphi_a\big|_{E_b}(e)+\varphi_a\big|_{E_b}(e')),\end{align*} which is continuous. For $r\in F$ and the scalar multiplication, the argument is similar.
\end{proof}

\colorbox{blue!30}{\textbf{Problem 1.}} Prove that the pair of maps $(p:E\to B,\sigma:B\to E)$ as defined above is a homotopic inverse pair, i.e. $p\circ\sigma\simeq\operatorname{Id}_B$ and $\sigma\circ p\simeq\operatorname{Id}_E$ (in fact, $p\circ\sigma=\operatorname{Id}_B$).
\begin{proof}
That $p\circ\sigma=\operatorname{Id}_B$ is clear from the definition of the zero section. That $\sigma\circ p\simeq\operatorname{Id}_E$ can be deduced by considering the homotopy \begin{align*}A:\sigma\circ p\leftrightarrow\operatorname{Id}_E:I\times E&\to E\\(t,e)&\mapsto t\cdot(\sigma\circ p)(e)+(1-t)\cdot e=(1-t)\cdot e,\end{align*} which is continuous since $E_{p(e)}$ is a topological vector space.
\end{proof}

\colorbox{red!30}{\textbf{Definition 2 (Morphism between Bundles).}} A \textbf{morphism} from vector bundle $\xi:E\overset{p}{\to}B$ to $\eta:E'\overset{p'}{\to}B$ is a continuous map $f:E\to E'$ such that \begin{enumerate}\item The diagram
$$\begin{tikzcd}
E \arrow[rd, "p"'] \arrow[rr, "f"] &   & E' \arrow[ld, "p'"] \\
                                   & B &
\end{tikzcd}$$
commutes;
\item For all $b\in B$, $f\big|_{E_b}$ is a map $E_b\to E'_{b}$ and it's a linear map.
\end{enumerate}
If $f$ above is a homeomorphism, then $f^{-1}$ is also a bundle morphism, in which case $\xi$ and $\eta$ are called \textbf{equivalent bundles}. Denote the set of all $n$-dimensional $F$-vector bundles modulo this equivalence relation with $\mathrm{Vect}_F^n(B)$.

\textbf{Definition 3 (Euclidean Metric).} Let $\xi:E\overset{p}{\to}B$ be an $n$-dimensional real vector bundle, then an \textbf{Euclidean metric} on $\xi$ is a continuous function $Q:E\to\mathbb{R}$ such that for all $b\in B$, $Q_b=Q\big|_{E_b}:E_b\to\mathbb{R}$ is a positive-definite quadratic form. Thus let \begin{align*}H_b:E_b\times E_b&\to\mathbb{R}\\(u,v)&\mapsto\frac{1}{2}(Q_b(u+v)-Q_b(u)-Q_b(v))\end{align*} be the associated bilinear form of the Euclidean metric: it is bilinear, symmetric, and non-degenerate, an inner product.

\colorbox{red!30}{\textbf{Proposition.}} If $B$ is paracompact and Hausdorff, then any real $n$-dimensional vector bundle $\xi:E\overset{p}{\to}B$ admits an Euclidean metric.
\begin{proof}
Define \begin{align*}f_n:\mathbb{R}^n&\to\mathbb{R}\\(x_1,\dots,x_n)&\mapsto x_1^2+\cdots+x_n^2.\end{align*} Let $\{(B_a,\varphi_a)\}_{a\in A}$ be a trivialization of the bundle. Define $\hat{Q}_a=f_n\circ\pi_2\circ\varphi_a$ in which $pi_2$ is the projection onto the second factor. Since $B$ is a paracompact and Hausdorff space, there exists (by the axiom of choice) a locally finite partition of unity subordinate to the open cover $\{B_a\}_{a\in A}$, i.e. $$\{\lambda_a:B\to[0,1]\,\big|\,a\in A\}.$$ From the partition of unity, an Euclidean metric is defined to be \begin{align*}Q:E&\to R\\e&\mapsto\sum_{a\in\{a\in A\,|\,p(e)\in B_a\}}\lambda_a(p(e))\cdot\hat{Q}_a(e).\end{align*}
\end{proof}

\subsection{Examples of Vector Bundle}
\textbf{Example 1.} Let $M$ be an $n$-dimensional $C^\infty$ manifold. Its tangent bundle $(TM, M, p)$ is a real $n$-dimensional vector bundle on $M$. An Euclidean metric thereof is also called a \textbf{Riemannian metric}.

\textbf{Example 2.} Let $M\subseteq\mathbb{R}^m$ an $n$-dimensional $C^\infty$ submanifold. For any $x\in M$, let $M_x$ be the tangent plane of $M$ at $x$ ($\dim M_x=n$) and $E_x$ be the normal plane ($\dim E_x=m-n$), then we have (1) the tangent bundle $TM=\bigsqcup_{x\in M}M_x$ and the normal bundle $\gamma(M)=\bigsqcup_{x\in M}E_x$ with base space $M$ and the obvious projections. The \textbf{first fundamental form} of $M$ is defined as $I:TM\to\mathbb{R}:v\mapsto\lVert v\rVert^2$.

\textbf{Example 3.} On $B$, $1$-dimensional real (resp. complex) vector bundles are also called real (resp. complex) line bundles. Consider the following two real line bundles on $B=S^1$. \begin{align*}
\xi&:E=[0,1]\times\mathbb{R}\big/(0,v)\sim(1,v)\overset{p}{\to}S^1;\\
\eta&:E'=[0,1]\times\mathbb{R}\big/(0,v)\sim(1,-v)\overset{p}{\to}S^1.
\end{align*}
$E$ is homeomorphic to the open cylinder and $E'$ to the open M\"obius stripe.

\colorbox{red!30}{\textbf{Definition S\counter (Orientation Bundle of a Vector Bundle).}} Assume we have a vector bundle $\eta:E\overset{p}{\to}B$ and a choice of trivialization $\{(U_a,\phi_a)\}_{a\in A}$. We define the \textbf{orientation bundle} (a fiber bundle, not a vector bundle) of $\eta$, $\hat{\eta}=\langle\hat{E},B,q,\pm1\rangle$ as follows. First, the underlying set of the total space is defined to be $$\hat{E}=\bigsqcup_{b\in B}\left(^{F_{\mathrm{GL}}(E_b)}\big/_{\mathrm{GL}^+}\right);$$ and the base space $\hat{B}=B$. Given an open set $B'\subseteq U_a$ for some $a$, define $$\mu_{B'}={\displaystyle\left(\bigsqcup_{b'\in B'}F_{\mathrm{GL}}(E_{b'})\right)}\Big/{\sim}$$ where $e\sim e'$ if and only if there exists $g^+\in\mathrm{GL}^+$ such that $(\varphi_a(e))_2\cdot g^+=(\varphi_a(e'))_2$; for $b'\in B'$, define a pseudo-inclusion map \begin{align*}\psi^{B'}_{b'}:\mu_B&\to(b',-)\in\hat{\eta}\\\nu&\mapsto[(\beta,\cdots)\in\mu_{B'}\,\big|\,\beta=b']\end{align*} and define $U(\mu_{B'})$ to be the set of all $\mu_{b'}\in\hat{\eta}$ such that $b'\in B'$ and $\mu_{b'}=\psi^{B'}_{b'}(\mu_{B'})$. Topologizing $\hat{E}$ with the basis of the topology being the sets $U(\mu_{B'})$, indexed over all possible $B'$, a projection $q:\hat{\eta}\to B$ is just a projection into the first factor.

\colorbox{red!30}{\textbf{Lemma S\counter.}} The orientation bundle of any vector bundle is a two-sheeted covering space thereof.
\begin{proof}
Trivial since $$\left|^{F_{\mathrm{GL}}(E_b)}\big/_{\mathrm{GL}^+}\right|=|\mathrm{GL}:\mathrm{GL}^+|=2.$$
\end{proof}

\colorbox{red!30}{\textbf{Lemma S\counter (Orientability of Orientation Bundle).}} The orientation bundle of any vector bundle with connected and compact base space is orientable.
\begin{proof}
Choose a finite open cover $U_i$ and its corresponding family of trivialization maps $\varphi_i:p^{-1}(U_i)\to B\times F^n$. Consider the intersection graph of $U_i$, which is clearly connected since the base space itself is connected, thus if we have a procedure to glue $\mu_{B'_k}$ and $\mu_{B'_l}$ together, we will have a method to create two global sections of the orientation bundle, which results in an assignment of the $\pm1$ thereto that is continuous, as required by the orientability condition for fiber bundle, i.e. fiberwise orientation-preserving trivialization maps. And indeed we have: On the intersection of two sets $U_i,U_j$ in the open cover, two equivalence classes $[e]\in\mu_{U_i},[e']\in\mu_{U_j}$ are equivalent and to be merged if and only if $(\phi_i(e))_2\cdot g^+=(\phi_j(e'))_2$ for some $g^+\in\mathrm{GL}^+$, and we are done.
\end{proof}

\colorbox{red!30}{\textbf{Definition S\counter (Orientation of Vector Bundle).}} If the frame bundle $F(\xi)$ as a $\mathrm{GL}$-principal bundle has a reduction to a $\mathrm{GL}^+$-bundle, then the vector bundle $\xi$ is orientable.

\colorbox{red!30}{\textbf{Proposition S\counter (Criterion for Orientability of Vector Bundle).}} Let $\eta:E\overset{p}{\to}B$ a vector bundle with $B$ connected, then $\eta$ is orientable if and only if the total space $\hat{E}$ of the orientation bundle $\hat{\eta}$ has two connected components.
\begin{proof}
If $B$ is connected, $\hat{E}$ has either one or two component(s) since it's a two-sheeted covering space of $B$, If it has two, then they are each mapped homeomorphically to $B$ by the covering projection defined above, splitting the fibers into 2 classes: voil\`a, two reduction of the frame bundle from $\mathrm{GL}$ to $\mathrm{GL}^+$. Conversely, if $\eta$ is orientable, it has a reduction of the frame bundle of the kind aforementioned whose complement in the frame bundle is also such a reduction. These two reductions of the frame bundle correspond to the two components of the orientation bundle which are disconnected since the reductions themselves are disjoint, open in $\hat{E}$, and non-empty (for $\mathrm{GL}^+$ acts freely on them).
\end{proof}

\colorbox{red!30}{\textbf{Lemma S\counter.}} Trivial bundle is orientable.
\begin{proof}
Trivial.
\end{proof}

\colorbox{red!30}{\textbf{Remark S\counter.}} To prove unorientability, one proceed with constructing a path in the base space and a continuous function assigning to each point on the path an element of $\hat{E}$ (an element of the fiber of the frame bundle at that point suffices), then this is a path in $\hat{E}$ since the lifting of path in base space along continuous map is always a path in the covering space. This path, should the endpoints be in different sheets, then connects the two sheets and gives the unorientability.

\colorbox{blue!30}{\textbf{Problem 2.}} Prove that $\xi\not\simeq\eta$.
\begin{proof}
Referring to Remark S6, we give a path connecting the two sheets of the covering space $\hat{E}$ at point $\frac{\pi}{2}$. The path is as follow: the end points are both $(0,1)$ in the base space and $[\frac{\pi}{2},1],[\frac{\pi}{2},-1]$ in the covering space, going counterclockwise once on the base space, assigning $[b,1]$ on $\frac{\pi}{2}<b<1$, $[1,1]=[0,-1]$ on $b=0$, and $[b,-1]$ on $0<b<\frac{\pi}{2}$. On the other hand, $\xi$ is isomorphic to the trivial bundle, which is orientable. Being an invariant of vector bundle morphism, orientability distinguishes these two bundles up to isomorphism.
\end{proof}

\textbf{Example 4 (Hopf Line Bundle).} Let $F\mathbb{P}^{n-1}$ be the space of lines in ${F}^n$ passing through the origin; this is called the $(n-1)$-dimensional $F$-projective space. Let $$E=\{(l,v)\in F\mathbb{P}^{n-1}\times F\,\big|\,v\in l\}\quad\quad\begin{aligned}p:E&\to F\mathbb{P}^{n-1}\\(l,v)&\mapsto l\end{aligned},$$ then this forms an $F$-line bundle on $F\mathbb{P}^{n-1}$, which is called the Hopf line bundle on $F\mathbb{P}^{n-1}$, denoted as $\gamma_F^{n-1}$. Specifically, \begin{itemize}\item $\gamma_\mathbb{R}^{n-1}$ is the real Hopf line bundle on $\mathbb{RP}^{n-1}$;$\dim_{\mathbb{R}}\gamma_\mathbb{R}^{n-1}=1$;\item $\gamma_\mathbb{C}^{n-1}$ is the complex Hopf line bundle on $\mathbb{CP}^{n-1}$;$\dim_{\mathbb{R}}\gamma_\mathbb{C}^{n-1}=2$;\item $\gamma_\mathbb{H}^{n-1}$ is the quaternionic Hopf line bundle on $\mathbb{HP}^{n-1}$;$\dim_{\mathbb{R}}\gamma_\mathbb{H}^{n-1}=4$.\end{itemize}

\colorbox{red!30}{\textbf{Fact S\counter (Sphere Bundle over $F\mathbb{P}^n$).}} In this segment we assume that $F=\mathbb{R},\mathbb{C},\mathbb{H}$. Let $|\cdot|$ denote the the modulus in $F$. For $x=(x_1,\dots,x_n)\in F^n$, define the Euclidean quadratic form $\lVert x\rVert=\sum_{i=1}^n|x_i|^2$ and the sphere $S^n(F)=\{x\in F^{n+1}\,\big|\,\lVert x\rVert=1\}$ with respect to it. Note that a group structure can be given to $S^0(F)$ since it's the kernel of the modulus homomorphism $F\setminus\{0\}\to(\mathbb{R},\cdot)$ Recall the definition of the $n$-dimensional $F$-projective space $$F\mathbb{P}^n=^{(F^{n+1}\setminus\{0\})}\big/_{F\setminus\{0\}}$$ where the quotient is via the diagonal left multiplication action and the quotient map thereof is denoted $\pi$. Now there is a $0$-sphere bundle $\chi:S^n(F)\overset{p}{\to}F\mathbb{P}^n$ with fiber $S^0(F)$ where $$p=\pi\circ i:S^n(F)\xhookrightarrow{}F^{n+1}\setminus\{0\}\to F\mathbb{P}^n$$ where the $i$ is the inclusion map.

\colorbox{red!30}{\textbf{Fact S\counter (CW structure of $\mathbb{CP}^{n}$).}} Now we describe one of the possible structure of $\mathbb{CP}^{n}$, a $2n$-dimensional real manifold, as a CW complex: First note that $\mathbb{CP}^{n}=S^{2n+1}/S^1$. Given a point in $\mathbb{CP}^{n}$ with homogeneous coordinates $\vec{z}=(z_0,\dots,z_{n})\in\mathbb{C}^{n+1}$, let $\phi=-\arg(z_{n})$, then under the equivalence relation defining $\mathbb{CP}^{n}$, $$\frac{1}{|\vec{z}|}(e^{i\phi}z_0,\dots,e^{i\phi}z_{n-1},r)=\vec{z}$$ where $r=|z_{n+2}|\in[0,1)\subset\mathbb{C}$. Representatives $\vec{z}'$ of this form ($|\vec{z}'|=1$ and $z'_{n+2}\in[0,1]$) parameterize the real disk $D^{2n+3}$ with boundary the $(2n+2)$-sphere with coordinate constraint $r=0$. The resulting function $q:D^{2n+1}\to\mathbb{CP}^{n}$ is continuous since it can be factored as
\begin{align*}
q:D^{2n+1}&\hookrightarrow\mathbb{C}^{n+1}\setminus\{0\}&\overset{\pi}{\to}\mathbb{CP}^{n}\\
(\Re(z_0),\Im(z_0),\dots,\Re(z_{n-1}),\Im(z_{n-1}),r)&\mapsto(z_0,\dots,z_{n-1},r)&\mapsto[z_0,\dots,z_{n-1},r]&
\end{align*}
where the first map is the embedding of the disk as a hemisphere in $\mathbb{R}^{2n+1}\hookrightarrow\mathbb{R}^{2n+2}\simeq\mathbb{C}^{n+1}$ and the second the quotient map. The only remaining part of the action by $\mathbb{C}\setminus\{0\}$ which fixes the condition $|z'|=1$ and $z'_{n+2}\in[0,1)$ is the sphere $S^1\subset\mathbb{C}\setminus\{0\}$ acting on the elements with $r=0$ by phase shifts on the coordinates $w=(z_0,\dots,z_{n})$. Regarded as real coordinates, this parameterizes an $S^{2n}$, and thus the result from the previous segment can be applied: voil\`a, une fonction attachant $S^{2n}\to\mathbb{CP}^{n-1}$, which results in the CW structure of $\mathbb{CP}^{n}$ by induction.

\colorbox{blue!30}{\textbf{Problem 3.}} Prove that $\gamma^n_\mathbb{R}\not\simeq\mathbb{RP}^{n-1}\times\mathbb{R}$ and that  $\gamma^n_\mathbb{C}\not\simeq\mathbb{CP}^{n-1}\times\mathbb{C}$.
\begin{proof}
Similar to Problem 2, starting and ending at $$A=\{x\in\mathbb{R}^{n}\,\big|\,x_1\in\mathbb{R},x_2=0,\dots,x_{n}=0\}$$ in the base space and $(A,1),(A,-1)$ respectively in the covering space, the path goes counterclosewise in the $X_1OX_2$ plane, with the continuous function assigning $(b,1)$ until is goes to $(A,-1)$. With the complex Hopf line bundle, this doesn't work anymore since it is orientable:. To that end, to disprove the isomorphism, we need to explicitly compute the homology groups of the two. The cellular chain complex of $\mathbb{CP}^n$, according to the previous segment, is $\mathbb{Z}$ at even indices and $0$ at odd indices, resulting in the homology groups $\mathbb{Z}$ at even indices and $0$ at odd indices.
\end{proof}

\textbf{Example 5 (Canonical Bundle on Grassmannian Manifold).} Let $\mathrm{Gr}_{n,k}(F)$ be the space of $k$-dimensional subspaces in $F^n$. The \textbf{Grassmannian Manifold} is defined to be \begin{itemize}\item\textbf{Real Grassmannian Manifold:} $\mathrm{Gr}_{n,k}(\mathbb{R})=\displaystyle\frac{O^+(n)}{O^+(k)\times O^+(n-k)}$;\item\textbf{Complex Grassmannian Manifold:} $\mathrm{Gr}_{n,k}(\mathbb{C})=\displaystyle\frac{U(n)}{U(k)\times U(n-k)}$;\item\textbf{Quaternionic Grassmannian Manifold:} $\mathrm{Gr}_{n,k}(\mathbb{H})=\displaystyle\frac{Sp(n)}{Sp(k)\times Sp(n-k)}$.\end{itemize} Let $$E_{n,k}=\{(l,v)\in \mathrm{Gr}_{n,k}(F)\times F^n\,\big|\,v\in l\}\quad\quad\begin{aligned}p:E_{n,k}&\to\mathrm{Gr}_{n,k}(F)\\(l,v)&\mapsto l\end{aligned}.$$ $(E_{n,k},\mathrm{Gr}_{n,k}(F),p)$ is a $k$-dimensional $F$-vector bundle on $\mathrm{Gr}_{n,k}(F)$, denoted as $\gamma_{n,k}(F)$ and read as the \textbf{tautological bundle} on $\mathrm{Gr}_{n,k}(F)$; and that $$\dim_F\gamma_{n,k}(F)=k\quad\dim_{\mathbb{R}}\gamma_{n,k}(\mathbb{R})=k\quad\dim_{\mathbb{R}}\gamma_{n,k}(\mathbb{C})=2k\quad\dim_{\mathbb{R}}\gamma_{n,k}(\mathbb{H})=4k.$$

\subsection{Constructs of Vector Bundle}
In this section, vector bundle means real vector bundle; there exists a completely parallel constructs for the complex ones.

\begin{enumerate}
\item \textbf{Restriction:} With known vector bundle $\xi:E\overset{p}{\to}B$ and subspace $X\subseteq B$, and $p\big|_X:p^{-1}(X)\to X$ being the restriction map, the triple $\xi\big|_X=(p^{-1}(X),X,p)$ is a vector bundle, called the \textbf{restriction bundle of $\xi$ on $X$}.
\item \textbf{Product:} With two known vector bundles $\xi:E\overset{p}{\to}B$ and $\eta:E'\overset{p'}{\to}B$, the product map $p\times p':E\times E'\to B\times B'$ is a vector bundle on the product space $B\times B'$, denoted as $\xi\times\eta$ and read as the \textbf{product bundle} of $\xi$ and $\eta$.
\item \colorbox{red!30}{\textbf{Pullback:}} With known vector bundle $\xi:E\overset{p}{\to}B$ and a continuous map $f:X\to B$, let $$\begin{aligned}p_f:E_f=\{(x,e)\in X\times E\,\big|\,e\in E_{f(x)}\}&\to X\\(x,e)&\mapsto x\end{aligned}\quad\text{ and }\quad\begin{aligned}\hat{f}:E_f&\to E\\(x,e)&\mapsto e\end{aligned},$$ then the diagram
$$\begin{tikzcd}
E_f \arrow[dd, "p_f"'] \arrow[rr, "\hat{f}"] &  & E \arrow[dd, "p"] \\
                                             &  &                   \\
X \arrow[rr, "f"']                           &  & B
\end{tikzcd}$$
commutes. In the diagram, \begin{enumerate}\item $f^*\xi:E_f\overset{p_f}{\to}X$ is a vector bundle on $X$ which is called the pullback bundle of $f$;\item The induced map $\hat{f}$ is a bundle morphism.\end{enumerate}
\item \textbf{Direct Sum (Coproduct):} With two known vector bundles on the same space $B$, $\xi:E\overset{p}{\to}B$ and $\eta:E'\overset{p'}{\to}B$, let \begin{align*}\Delta:B&\to B\times B\\b&\mapsto(b,b)\end{align*} the diagonal embedding, then the pullback bundle $\Delta^*(\xi\times\eta)$ is called the \textbf{direct sum (coproduct)}, denoted as $\xi\oplus\eta$.
\end{enumerate}

\colorbox{red!30}{\textbf{Remark S\counter.}} The category of $n$-dimensional $F$-vector bundles, $\mathrm{Vect}_n(F)$ is not preabelian, since the kernel of a morphism from the trivial line bundle on $\mathbb{R}$ to itself can have a "jump of rank" (with the language of Vakil) and thus have no kernel. On the other hand, according to the constructs aforementioned, this category is additive, and thus an important consequence is that the finite products and coproducts coincide.

\begin{enumerate}[resume]
\item \textbf{Subbundle and Complement Bundle:} With known vector bundle $\xi:E\overset{p}{\to}B$, if $E$ has a subspace $E'\subseteq E$ such that the composition $q:E'\overset{i}{\to}E\overset{p}{\to}B$ is another vector bundle, then $\eta=(E',B,q)$ is called a \textbf{subbundle} of $\xi$, denoted $\eta\subseteq\xi$. If $\eta\subseteq\xi$, then a vector bundle $\mu$ such that $\eta\oplus\mu=\xi$ is called the \textbf{complement bundle} of the subbundle $\eta$.
\end{enumerate}

\colorbox{red!30}{\textbf{Proposition S\counter (Existence of Complement Bundle).}} Complement bundle $\mu$ exists for every pair $\eta\subseteq\xi$.
\begin{proof}
Each of the fiber on a bundle can be made into an inner product space by the proposition under Definition 3. Since these fibers are finite dimensional the subspaces thereof are always closed. Take the orthogonal complement of each fiber of $E'_b\subseteq E_\eta$ with respect to the corresponding $E_b\subseteq E_\xi$ and let it be $E^\perp_b\subseteq E_\mu$ and we have $E'_b\oplus E^\perp_b=E_b$ by the property of the orthogonal complement of subspace of inner product space; with the appropriate structures adjoined, $\mu$ becomes a vector bundle that happens to be the complement bundle desired.
\end{proof}

\newcommand{\Hom}{\operatorname{Hom}}
\begin{enumerate}[resume]
\item \textbf{$\mathbf{Hom}$ bundle:} With two known vector bundles $$\xi:E\overset{p}{\to}B\quad\text{ and }\quad\eta:E'\overset{p'}{\to}B,$$ let $$\Hom(\xi,\eta):\Hom(E,E')=\bigsqcup_{b\in B}\Hom(E_b,E'_b)\to B,$$ then it's a vector bundle and $\dim\Hom(\xi,\eta)=\dim\xi\cdot\dim\eta$.
\end{enumerate}

\colorbox{red!30}{\textbf{Proposition (Properties of $\mathbf{Hom}$ bundle).}} \begin{enumerate}
\item
\item
\item
\end{enumerate}
\begin{proof}

\end{proof}

\subsection{Tangent Bundle of $S^{n-1}$ and $F\mathbb{P}^n$}

\section{The Thom Isomorphism Theorems}

Because of his establishment of the cobordism theory, Ren\'e Thom was awarded with the FIELDS medal in 1956. To that end, he made an important communicative result between the (co)homology theory of manifold, differential topology, and the theory of characteristic classes, in which the most fundamental, practical, and influential one is the Thom isomorphism theorem which will be introduced in this section.

Let $\xi:E\overset{p}{\to}B$ be an $n$-dimensional real vector bundle. The zero section $\sigma:B\to E:b\mapsto0\in E_b$ defines a subspace of $E$,
$$E^{\neq0}=E\setminus\operatorname{Im}\sigma=\bigsqcup_{b\in B}(E_b\setminus\{0\}).$$
And thus we obtain a pair of space $(E,E^{\neq0})$; on the other hand, for any point $b\in B$, the inclusion map of fibre determines a map of pair of space
$$i_b:(E_b,E_b\setminus\{0\})\to(E,E^{\neq0}).$$
Consider the cohomology isomorphism induced by it, i.e.
$$i_b^*:H^n(E,E^{\neq0};G)\leftrightarrow H^n(E_b,E_b\setminus\{0\};G),$$
in which $G=(\mathbb{Z},+)$ or $G=(\mathbb{Z}/2\mathbb{Z},+)$.
Note also that by the long exact sequence of relative cohomology and the facts that $H^r(E_b;G)=H^r(\mathbb{R}^n;G)=0$ and that $H^r(E_b\setminus\{0\};G)=H^r(\mathbb{R}^n\setminus\{0\}\simeq S^{n-1};G)$ by the homotopy invariance of singular cohomology,
$$H^r(E_b,E_b\setminus\{0\};G)=H^{r-1}(S^{n-1};G)=\begin{cases}G&r=n\\0&\mbox{otherwise}\end{cases}.$$

\subsection{Mod-2 Thom Isomorphism Theorem}

In this subsection, assume the coefficient group of (co)homology is $\mathbb{Z}/2$.

\colorbox{red!30}{\textbf{Theorem S\counter (Algebraic K\"unneth Exact Sequence).}} Let $C_\bullet,D_\bullet$ be chain complexes over a principal ideal domain $R$, and let $C_n,D_n$ be free $R$-modules for each $n$, then there exists a natural exact sequence, called the K\"unneth exact sequence,
$$
0\to\bigoplus_{p+q=n}H_p(C_\bullet)\otimes H_q(D_\bullet)\overset{\times}{\to}H_n(C_\bullet\otimes D_\bullet)\to\bigoplus_{p+q=n}\mathrm{Tor}^R_1(H_p(C_\bullet),H_{q-1}(D_\bullet))\to0
$$
where the cross is defined to be the unique natural, bilinear, and normalized map
\begin{align*}
\times:H_p(C_\bullet)\otimes H_q(D_\bullet)&\to H_{p+q}(C_\bullet\otimes D_\bullet).
\end{align*}
and the penultimate map is constructed in the proof.

\begin{proof}
Let $Z_p=\ker\partial:C_p\to C_{p-1},B_p=\operatorname{im}\partial:C_{p+1}\to C_p$ and obtain the short exact sequence
$$0\to Z_p\to C_p\overset{\partial}{\to}B_{p-1}\to0$$
which may be viewed as a short exact sequence of free chain complexes by giving $Z_*$ and $B_*$ the zero differentials (the submodule $Z_p$ and $B_p$ of free module $C_p$ over principal ideal domain $R$ is free\footnote{Martin Brandenburg (https://math.stackexchange.com/users/1650/martin-brandenburg), Submodule of free module over a p.i.d. is free even when the module is not finitely generated?, URL (version: 2012-06-25): https://math.stackexchange.com/q/162958}).
Since $B_p$ is free, applying the tensor functor to the short exact sequence aformentioned with $D_p$ and $D_{p-1}$ respectively yields a new exact sequence of chain complexes, i.e.
$$B_\bullet\otimes D_\bullet\overset{\mu_\bullet}{\to} Z_\bullet\otimes D_\bullet\overset{\nu_\bullet}{\to} C_\bullet\otimes D_\bullet\to0$$
by the fourth axiom of derived functor.
Since the differential in the complex $Z_\bullet$ is zero, we may compute the Leibnitz rule:
\begin{align*}
\partial_{p,q}:Z_p\otimes D_q&\to Z_{p-1}\otimes D_{q-1}\\z\otimes d&\mapsto(-1)^{p}z\otimes\partial_q d.
\end{align*}
Also note that since $Z_p$ are free, they are flat also.\footnote{Jesko Hüttenhain (https://math.stackexchange.com/users/11653/jesko-h\%c3\%bcttenhain), Proving that free modules are flat (without appealing projective modules), URL (version: 2013-05-31): https://math.stackexchange.com/q/407423}
Compute the chain complex by definition of tensor of complex and the homology
$$(Z_\bullet\otimes D_\bullet)_n=\bigoplus_{p+q=n}(Z_p\otimes D_q)\quad\quad H_n(Z_\bullet\otimes D_\bullet)=\bigoplus_{p+q=n}(Z_p\otimes H_q(D_\bullet))=\bigoplus_{p+q=n}(H_p(Z_\bullet)\otimes H_q(D_\bullet))$$
from the fact that homology preserves coproduct (direct sum), and similarly
$$H_n(B_\bullet\otimes D_\bullet)=\bigoplus_{p+q=n}(B_p\otimes H_q(D_\bullet))=\bigoplus_{p+q=n}(H_p(B_\bullet)\otimes H_q(D_\bullet)).$$
Now apply the zig-zag lemma to the last short exact sequence and obtain an long exact sequence
$$
\begin{tikzcd}
                                                                         & \cdots                                        &                                               \\
H_m(B_\bullet\otimes D_\bullet) \arrow[r, "\iota_m"]                     & H_m(Z_\bullet\otimes D_\bullet) \arrow[r]     & H_m(C_\bullet\otimes D_\bullet) \arrow[lld]   \\
H_{m-1}(B_\bullet\otimes D_\bullet) \arrow[r, "\iota_{m-1}"] & H_{m-1}(Z_\bullet\otimes D_\bullet) \arrow[r] & H_{m-1}(C_\bullet\otimes D_\bullet)  \\
                                                                         & \cdots                                        &
\end{tikzcd}.
$$
Consider the short exact sequence $$0\to\operatorname{coker}\iota_n\to H_n(C_\bullet\otimes D_\bullet)\to\ker\iota_n\to0,$$ which is a direct result from the long exact sequence above.
Notice that $$\operatorname{im}\mu_n=\ker\nu_n=0$$ by the definition of kernel, and thus that $$\operatorname{coker}\iota_n=\bigoplus_{p+q=n}(Z_p\otimes H_{q}(D_\bullet)).$$
On the other hand, tensoring $H_q(D_\bullet)$ with the exact sequence
$$0\to B_p\to Z_p\to H_p(C_\bullet)\to0$$
and extending the sequence to the left by the means of derived functor, we have a long exact sequence
$$0\to\mathrm{Tor}^R_1(H_p(C_\bullet),H_q(D_\bullet))\to B_p\otimes H_q(D_\bullet)\to Z_p\otimes H_q(D_\bullet)\to H_p(C_\bullet)\otimes H_q(D_\bullet)\to0$$
which vanishes after the first derivative since $R$ is a principal ideal domain\footnote{Eric Wofsey (https://math.stackexchange.com/users/86856/eric-wofsey), Sufficient conditions for homology to be free module, URL (version: 2017-09-22): https://math.stackexchange.com/q/2439872} and then due to the fourth axiom of derived functor.
Direct summed over $p+q=n$, we have
$$0\to\bigoplus_{p+q=n}\mathrm{Tor}^R_1(H_p(C_\bullet),H_q(D_\bullet))\overset{\sigma}{\to}\bigoplus_{p+q=n}(B_p\otimes H_q(D_\bullet))\overset{\iota_n}{\to}\bigoplus_{p+q=n}(Z_p\otimes H_q(D_\bullet))\to\bigoplus_{p+q=n}(H_p(C_\bullet)\otimes H_q(D_\bullet))\to0.$$
Observe that $\ker\iota_n=\operatorname{im}\sigma$, which is $\operatorname{dom}\sigma=\bigoplus_{p+q=n}\mathrm{Tor}^R_1(H_p(C_\bullet),H_q(D_\bullet))$ since $\sigma$ is injective, which results in the sequence
$$0\to\bigoplus_{p+q=n}H_p(C_\bullet)\otimes H_q(D_\bullet)\overset{\times}{\to}H_n(C_\bullet\otimes D_\bullet)\to\bigoplus_{p+q=n}\mathrm{Tor}^R_1(H_p(C_\bullet),H_{q-1}(D_\bullet))\to0,$$
as desired.

\end{proof}

\colorbox{red!30}{\textbf{Remark on the Statement of the Previous Theorem.}} The above sequence splits non-naturally, and also works for non-free $D_n$; the reader should consult \textit{A Course in Homological Algebra} by Peter Hilton and Urs Stammbach for further details. The aforementioned version of the algebraic K\"unneth formula also has a dual version which describes the property of cohomology, in which all chain complexes are substituted with cochain complexes and homologies with cohomologies, which will be the version used in the proof of the Thom isomorphism theorem.

\textbf{Theorem 1.} Let $\xi:E\overset{p}{\to}B$ be an $n$-dimensional real vector bundle where $B$ is paracompact. Then
\begin{enumerate}
\item There exists a unique cohomology class $U_2(\xi)\in H^n(E,E^{\neq0};\mathbb{Z}/2)$ such that for all $b\in B$, $i_b^*U_2(\xi)\in H^2(E_b,E_b\setminus\{0\};\mathbb{Z}/2)=\mathbb{Z}/2$ is the generator element;
\item For all $r\in\mathbb{Z}$, the homomorphism
\begin{align*}
T^r:H^r(B;\mathbb{Z}/2)&\to H^{n+r}(E,E^{\neq0};\mathbb{Z}/2)\\x&\mapsto p^*(x)\smile U_2(\xi)
\end{align*}
is an isomorphism of graded Abelian group.
\end{enumerate}

\textbf{Definition.} In Theorem 1, $U_2(\xi)$ is to be called the \textbf{Mod-2 Thom class}; and $T$ the \textbf{Mod-2 Thom isomorphisms} of $\xi$.

\colorbox{blue!30}{\textbf{Lemma 1.}} For trivial bundle $p:E=B\times\mathbb{C}^n\overset{\pi}{\to}B$, Theorem 1 holds.
\begin{proof}
By premises we have $(E,E^{\neq0})=(B,\emptyset)\times(\mathbb{C}^n,\mathbb{C}^n\setminus\{0\})$. Define the relative cochain complex $C^i=S^i(B,\emptyset;\mathbb{Z}/2)$ and $D^i=S^i(\mathbb{C}^n,\mathbb{C}^n\setminus\{0\};\mathbb{Z}/2)$. Applying the algebraic K\"unneth formula on these complexes and noticing
$$\mathrm{Tor}^{\mathbb{Z}/2}_1(H^p(C^\bullet),H^q(D^\bullet))=0$$
from $\mathbb{Z}/2$ being a field results in the computation
$$H^n(E,E^{\neq0};\mathbb{Z}/2)=H^n((B,\emptyset)\times(\mathbb{C}^n,\mathbb{C}^n\setminus\{0\});\mathbb{Z}/2)=H^0(B;\mathbb{Z}/2)\otimes H^n(\mathbb{C}^n,\mathbb{C}\setminus\{0\},\mathbb{Z}/2)=H^0(B;\mathbb{Z}/2)\otimes\mathbb{Z}/2$$
since for $0\leq i<n$, $H^i(\mathbb{C}^n,\mathbb{C}\setminus\{0\},\mathbb{Z}/2)=0$ by the long exact sequence of relative singular cohomology.
Let $m$ be the number of path connected components of $B=\bigsqcup_{1\leq i\leq m}B_i$ in which $B_i$ are the components, then
$$H^0(B;\mathbb{Z}/2)=\bigoplus_{1\leq i\leq m}H^0(B_i)=\bigoplus_{1\leq i\leq m}\mathbb{Z}/2[1_{B_i}]$$
in which $1_{B_i}\in H^0(B_i)\simeq\mathbb{Z}/2$ is the identity, the generator, and the sole nonzero element of the group.
Notice that $1_B=\bigoplus_{1\leq i\leq m}1_{B_i}\in H^0(B;\mathbb{Z}/2)$ is also the identity, the generator, and the sole nonzero element of the zeroth cohomology group of $B$.
Let $\omega$ denote such an element\footnote{the identity, the generator, and the sole nonzero element thereof} of $H^n(\mathbb{C}^n,\mathbb{C}^n\setminus\{0\};\mathbb{Z}/2)\simeq\mathbb{Z}/2$,
and let
$$U_2(\xi)=1_B\otimes\omega\in H^n(E,E^{\neq0}).$$
Now we have
\begin{enumerate}
\item For any $b\in B$, $$i_b^*U_2(\xi)\in H^n(E_b,E_b\setminus\{0\})$$ is a generator;
\item By algebraic K\"unneth formula (again the Tor vanishes since $\mathbb{Z}/2$ is a field),
$$H^{(r=\dim B)+n}(E,E^{\neq0})=\bigoplus_{p+q=r+n}H^p(B)\otimes H^q(\mathbb{C}^n,\mathbb{C}^n\setminus\{0\})=H^r(B)\otimes H^n(\mathbb{C}^n,\mathbb{C}\setminus\{0\}).$$
For any $y\in H^{r+n}(E,E^{\neq0})$, it can be decomposed into tensor factors $$y=x\otimes\omega=(x\otimes 1_{\mathbb{C}^n})\smile(1_B\otimes\omega)=p^*(x)\smile U_2(\xi),$$ where the last equality comes from the computation
$$\begin{aligned}(p:(E,E^{\neq0})\to B)^*(x\in H^r(B))&(h\overset{(\times)^{-1}}{\leftrightarrow}j\otimes k\in H_0(B)\otimes H^r(\mathbb{C}^n,\mathbb{C}^n\setminus\{0\})\simeq H_r(E,E^{\neq0}))\\&=x(p_*((\times)\circ(j\otimes k)))\end{aligned}$$ where $\times$ is the homology map induced by the Eilenberg-Zilber map.
Recall the process of constructing the homology cross product using the Eilenberg-Zilber map, specifically, the shuffle product of singular simplex; let us denote it
$$\begin{aligned}\Phi:(\Delta^0\to B)\times\left(\Delta^r\to\frac{\mathbb{C}^n}{\mathbb{C}^n\setminus\{0\}}\right)&\to\left(\Delta^r\to\frac{E}{E^{\neq0}}\right)\\
(\alpha,\beta)&\mapsto\alpha\circ\Psi_1\times\beta\circ\Psi_2\end{aligned}$$
where $P_1$ and $P_2$ are the prism subdivision of standard $r$-simplex into standard $0$-simplex and $r$-simplex.
Let us compose this map with $p$ on the right, and this gives a map
$$\begin{aligned}p\circ\Phi:(\Delta^0\to B)\times\left(\Delta^r\to\frac{\mathbb{C}^n}{\mathbb{C}^n\setminus\{0\}}\right)&\to\left(\Delta^r\to\frac{B}{E^{\neq0}}=B\right)\\
(\alpha,\beta)&\mapsto p\circ(\alpha\circ\Psi_1\times\beta\circ\Psi_2)=\alpha\circ\Psi_1\end{aligned}.$$
If $j=[\iota]$, then $$x(p_*((\times)\circ(j\otimes k)))=\begin{cases}x([\iota\circ\Psi_1])&k\neq0\\0&\mbox{otherwise}\end{cases},$$ but this is exactly $(x\otimes 1_{\mathbb{C}^n})((\times)^{-1}(h)=[\iota\circ\Psi_1]\otimes l)$ where $l\in H_r(\mathbb{C}^n,\mathbb{C}^n\setminus\{0\})$.
\end{enumerate}
\end{proof}

\textbf{Lemma 2.} If Theorem 1 holds for bundle $\xi:E\overset{p}{\to} B$, and let $A$ be a subspace of $B$, and let $\xi\big|_{A}$ be a $2n$-dimensional complex trivial bundle, then there exists a Thom class for $\xi\big|_A$.
\begin{proof}
Let $I:E\big|_A\to E$ be the total space inclusion map, and it induces a map of pair of space $\hat{I}:(E\big|_A,(E\big|_A)^{\neq0})\to(E,E^{\neq0})$. Notice that the following diagram commutes:
$$
\begin{tikzcd}
                                                       & {(E_a,E_a\setminus\{0\})} \arrow[rd, "i_a"] \arrow[ld, "i_a"'] &                 \\
{(E\big|_A,(E\big|_A)^{\neq0})} \arrow[rr, "\hat{I}"'] &                                                                & {(E,E^{\neq0})}
\end{tikzcd}.
$$
Observe the induced cohomology homomorphism
$$\hat{I}^*:H^{2n}(E,E^{\neq0})\to H^{2n}(E\big|_A,(E\big|_A)^{\neq0}).$$
Take the singular cohomology functor on the previous diagram, we have
$$
\begin{tikzcd}
                                                          & {H^{2n}(E_a,E_a\setminus\{0\})} &                                                                    \\
{H^{2n}(E\big|_A,(E\big|_A)^{\neq0})} \arrow[ru, "i_a^*"] &                                 & {H^{2n}(E,E^{\neq0})} \arrow[lu, "i_a^*"'] \arrow[ll, "\hat{I}^*"]
\end{tikzcd}.
$$
Consider the Thom class $U(\xi)$, and the pullbacked class $\hat{I}^*U(\xi)\in H^{2n}(E\big|_A,(E\big|_A)^{\neq0})$, which satisfies that for all $a\in A\subseteq B$, $i_a^*[\hat{I}^*U(\xi)]=i_a^*U(\xi)\in H^{2n}(E_a,E_a\setminus\{0\})=\mathbb{Z}$ by the commutativity of the diagram; this concludes that $\hat{I}^*U(\xi)$ is the Thom class of $\xi\big|_A$.
\end{proof}

\begin{proof}[Proof to Theorem 1]
Let $\xi:E\overset{p}{\to}B$ be an $n$-dimensional complex bundle. Take an open cover $B=\bigcup_{\alpha\in\Lambda}B_\alpha$ such that $\xi_\alpha=\xi\big|_{B_\alpha}$ can be trivialized. Proceed with induction on $|\Lambda|$. The base case $|\Lambda|=1$ is trivial from Lemma 1. Let the induction hypothesis hold for $|\Lambda|=k-1$, i.e., for any bundle with trivializing open cover of cardinality $k-1$, Theorem 1 holds. Let us consider a bundle with cover indexed by $\Lambda=\{\alpha_1,\dots,\alpha_k\}$. Define
$$B_1=B_{\alpha_1}\cup\cdots\cup B_{\alpha_{k-1}}\quad\quad B_2=B_{\alpha_k}\quad\quad B_3=B_1\cap B_2\c.$$ Then the theorem holds for all of them by respectively the induction hypothesis, the Lemma 1, and the Lemma 2; let $U(\xi_i)\in H^{2n}(E_i=p^{-1}(B_i),E_i^{\neq0})$ be the corresponding Thom classes for $1\leq i\leq 3$, and let
\begin{align*}
T_i:H^r(B_i)&\to H^{2n+r}(E_i,E_i^{\neq0})\\
&\mapsto p_i^*(x)\smile U(\xi_i)
\end{align*}
be the corresponding Thom isomorphisms for $1\leq i\leq 3$.
Consider the inclusion maps of spaces
$$
\begin{tikzcd}
                                                     & E_1 \arrow[rd, "J_1"]  &               \\
E_3=E_1\cap E_2 \arrow[ru, "I_1"] \arrow[rd, "I_2"'] &                        & E=E_1\cup E_2 \\
                                                     & E_2 \arrow[ru, "J_2"'] &
\end{tikzcd},
$$
which induces a Mayer-Vietoris exact sequence of cohomology
$$\cdots\to H^{2n-1}(E_3,E_3^{\neq0})\to H^{2n}(E,E^{\neq0})\to H^{2n}(E_1,E_1^{\neq0})\oplus H^{2n}(E_2,E_2^{\neq0})\to H^{2n+1}(E_3,E_3^{\neq0})\to\cdots.$$
Since
$$I_1^*-I_2^*(U(\xi_1)\oplus U(\xi_2))=U(\xi_3)-U(\xi_3)=0$$
by Lemma 2, the image of the map $H^{2n}(E,E^{\neq0})\to H^{2n}(E_1,E_1^{\neq0})$ in the aforementioned sequence necessarily contains $U(\xi_1)\oplus U(\xi_2)$, which results in a $U\in H^{2n}(E,E^{\neq0})$ such that $(J_1^*,J_2^*)(U)=U(\xi_1)\oplus U(\xi_2)$, and thus $$i_b^*U=\left(\begin{rcases}\begin{cases}J_1\circ i_b&b\in B_1\\J_2\circ i_b&b\in B_2\end{cases}\end{rcases}\right)^*U=\begin{rcases}\begin{cases}i_b^*\circ J_1^*(U)=i_b^*U(\xi_1)&b\in B_1\\i_b^*\circ J_2^*(U)=i_b^*U(\xi_2)&b\in B_2\end{cases}\end{rcases}\in H^{2n}(E,E^{\neq0})=\mathbb{Z}$$ is a generator.
Since $H^{2n-1}(E_3,E_3^{\neq0})=0$, the map $J_1^*,J_2^*$ is injective, which guarantees the uniqueness of class $U$ which satisfies the above condition. This concludes the first part of Theorem 1.
\end{proof}

\newpage

Let $\xi:E\overset{p}{\to}B$ be an $n$-dimensional complex vector bundle. The zero section $\sigma:B\to E:b\mapsto0\in E_b$ defines a subspace of $E$,
$$E^{\neq0}=E\setminus\operatorname{Im}\sigma=\bigsqcup_{b\in B}(E_b\setminus\{0\}).$$
And thus we obtain a pair of space $(E,E^{\neq0})$; on the other hand, for any point $b\in B$, the inclusion map of fibre determines a map of pair of space
$$i_b:(E_b,E_b\setminus\{0\})\to(E,E^{\neq0}).$$
Consider the cohomology homomorphism induced thereby, i.e.
$$i_b^*:H^n(E,E^{\neq0};\mathbb{Z})\leftrightarrow H^n(E_b,E_b\setminus\{0\};\mathbb{Z}).$$
Note also that by the long exact sequence of relative cohomology and the facts that $H^r(E_b;G)=H^r(\mathbb{R}^n;\mathbb{Z})=0$ and that $H^r(E_b\setminus\{0\};\mathbb{Z})=H^r(\mathbb{R}^n\setminus\{0\}\simeq S^{n-1};\mathbb{Z})$ by the homotopy invariance of singular cohomology,
$$H^r(E_b,E_b\setminus\{0\};\mathbb{Z})=H^{r-1}(S^{n-1};\mathbb{Z})=\begin{cases}\mathbb{Z}&r=n\\0&\mbox{otherwise}\end{cases}.$$

\colorbox{red!30}{\textbf{Fact S\counter (Long Exact Sequence of Triple).}}
Let $A,B,C$ be a triple $(C;B,A)$, i.e., satisfying the condition $A\subseteq B\subseteq C$, then there is a long exact sequence of relative cohomology
$$\cdots\to H^{n-1}(B,A)\to H^n(C,B)\to H^n(C,A)\to H^n(B,A)\to H^{n+1}(C,B)\to\cdots.$$

\colorbox{red!30}{\textbf{Fact S\counter (Mayer-Vietoris Sequence for Cohomology).}}
For a triad $(X,A,B)$ with condition $X\subseteq\operatorname{int}A\cup\operatorname{int}B$, the following sequence is exact:
$$\cdots\to H^n(X)\to H^n(A)\oplus H^n(B)\to H^n(A\cap B)\to H^{n+1}(X)\to\cdots.$$

\colorbox{red!30}{\textbf{Theorem S\counter.}}
Let $\xi:E\overset{p}{\to}B$ be an $n$-dimensional complex vector bundle with the base space admitting a finite-dimensional cellular decomposition. If the Thom class exists, i.e., there exists a cohomology class $U(\xi)\in H^{2n}(E,E^{\neq0})$ such that for all $b\in B$, the pullback $i_b^*U(\xi)\in H^{2n}(E_b,E_b^{\neq0})=\mathbb{Z}$ is a generator, then the homomorphism
\begin{align*}
T^r:H^r(B)&\to H^{n+r}(E,E^{\neq0})\\
h_b&\mapsto p^*(h_b)\smile U(\xi),
\end{align*}
where the cup is the relative cup product
$$\smile:H^*(E)=H^r(E,\emptyset)\times H^{2n}(E,E^{\neq0})\to H^{r+2n}(E,\emptyset\cup E^{\neq0}=E^{\neq0}),$$
is a isomorphism of graded abelian groups.
\begin{proof}
Construct a new bundle $\eta:\hat{E}\overset{q}{\to}B$ by attaching the mapping cylinder
$$M={}^{((I\times E^{\neq0})\sqcup B)}\big/_{(0,e)\sim p(e)}$$
to $E$ by identifying the subspace $E^{\neq0}\subseteq E$ and $\{1\}\times E^{\neq0}\subseteq M$. The fibers of bundle $\eta$ are the cones
$$\begin{aligned}\hat{F}&={}^{((I\times \mathbb{C}^n\setminus\{0\}))\sqcup\mathbb{C}^n}\big/_{\langle(1,z)\sim(1,z'),(0,x)\sim y\,|\,z,x\in\mathbb{C}^n\setminus\{0\},y\in\mathbb{C}^n,x=y\rangle}
\end{aligned}$$
with no apparent vector space structure.
Regarding $B$ as a subspace of $\hat{E}$ at the $0$ end of $M$, we may compute
$$H^*(\hat{E},M)=H^*(\hat{E}\setminus B,M\setminus B)=H^*(E,E^{\neq0})$$ via excision and the homotopy equivalence (deformation retraction) between $\hat{E}\setminus B$ and $E$ (crushing the punctured complex spaces towards the direction of $1$, i.e., of $E$). The long exact sequence of triple $(\hat{E};M,B)$ with entry $H^{2n}(M,B)=0$ by homotopy equivalence between $M$ and $B$ (another deformation retraction, this time crushing towards the direction of $0$, i.e, of $B$) gives $H^*(\hat{E},M)=H^*(\hat{E},B)$. We may consider the aforementioned cohomologies as having structures of a graded $H^*(B)$-module with a singleton generating set $\{U(\xi)\}$ as following:
$$
\begin{aligned}(\cdot):H^a(B)\times H^b(E,E^{\neq0})&\to H^{a+b}(E,E^{\neq0})\\(h_b,s\cdot U(\xi))&\mapsto s\cdot p^*(h_b)\smile U(\xi)\end{aligned}\quad\quad
\begin{aligned}(\cdot):H^a(B)\times H^b(\hat{E},M)&\to H^{a+b}(\hat{E},M)\\(h_b,h_{(e,m)})&\mapsto p^*(h_b)\smile h_{(e,m)}\end{aligned}\quad\quad\text{et cetera.}
$$
Thus the isomorphisms aforementioned are all graded $H^*(B)$-module isomorphisms.
Consider the long exact sequence of relative cohomology:
$$\cdots\to H^*(\hat{E},B)\to H^*(\hat{E})\overset{\iota}{\to}H^*(B)\to\cdots$$
and the projection $q:\hat{E}\to B$ which is a deformation retract. Consider the induced cohomology map by $q$, $q^*:H^*(B)\to H^*(\hat{E})$; composed with $\iota$, it would be the identity, which means that the map $\iota$ is surjective, and thus gives a cutting of the long exact sequence into short one. By the splitting lemma, the above sequence splits, i.e.,
$$H^*(\hat{E})=H^*(\hat{E},B)\oplus H^*(B),$$ and furthermore, this is a splitting of graded $H^*(B)$-module. Let $u\in H^{2n}(\hat{E})$ be such that it corresponds to the $U(\xi)\in H^{2n}(E,E^{\neq0})=H^{2n}(\hat{E},B)$ in the above splitting isomorphism.
Appealing to the cohomological Mayer-Vietoris sequence, we may use the fact that
$$\hat{F}=\left(CF={}^{((0,1]\times\mathbb{C}^n\setminus\{0\})}\big/_{(1,z)\sim(1,z')}\right)\cup(F=\{0\}\times\mathbb{C}^n)$$
to deduce that $H^*(\hat{F})=H^*(F)\oplus H^*(CF)$ since $H^*(F\cap CF)=0$.
Recall that the homeomorphism between fiber over $b$ and the fiber space for a vector bundle is a linear homeomorphism between topological vector spaces, i.e., $E_b\simeq\mathbb{C}^n$. So for each $b\in B$, by premise of the existence of Thom class, the restriction $i^*U(\xi)\in H^{2n}(E_b,E_b^{\neq0})=H^{2n}(\mathbb{C}^n,\mathbb{C}^n\setminus\{0\})$ is a generator, and thus by the application of the epic map in the exact sequence of relative cohomology, this generator can be mapped to (in another word, determines) a generator of $H^{2n}(\mathbb{C}^n)$; together with $1\in H^{2n}(CF)$, they form a basis for $H^{2n}(\hat{F})$.
The last step is to apply the Leray-Hirsch theorem (Hatcher p.432, Theorem 4D.1) to the fiber bundle $\hat{F}\to\hat{E}\overset{q}{\to}B$, which results in an isomorphism $\Phi:H^r(B)\otimes H^{2n}(\hat{F})\to H^{r+2n}(\hat{E})$
which establishes that $H^{r+2n}(\hat{E})$ is a free graded $H^r(B)$-module with basis $\{U(\xi),1\in H^{2n}(CF)\}$. Now we may conclude that $\{U(\xi)\}$ is a basis for the free $H^r(B)$-module $H^{r+2n}(E,E^{\neq0})$, and thus $T^r$ is a isomorphism, by the property of free cyclic module.
\end{proof}

\section{Euler Class of Vector Bundle and Gysin Exact Sequence}
\subsection{Fiber Bundle}
\subsection{Euler Class of Vector Bundle and its Properties}

Let $\xi:E\overset{p}{\to}B$ is a $2n$-dimensional complex vector bundle, the Thom class thereof being $U()\xi\in H^{2n}(E,E^{\neq0})$. Let $J:(E,\emptyset)\to(E,E^{\neq0})$ be a natural inclusion of pair of space; and let $\sigma:B\to E$ be the zero section.

\textbf{Definition.} The image of the Thom class $U(\xi)$ under the map
$$\psi:H^{2n}(E,E^{\neq0})\overset{J^*}{\to}H^{2n}(E)\overset{\sigma^*}{\to}H^{2n}(B)$$
is called the Euler class of $\xi$, denoted $e(\xi)\in H^{2n}(B)$.

\colorbox{red!30}{\textbf{Theorem 3 (Properties of Euler Class).}} Let $\xi$ and $\eta$ be two orientable complex vector bundle on $B$, and let $f:X\to B$ a continuous map, then the Euler class satisfies:
\begin{enumerate}
\item {} [Functoriality] $e(f^*\xi)=f^*(\xi)$;
\item {} [Whitney Sum Formula] $e(\xi\oplus\eta)=e(\xi)\smile e(\eta)$;
\item {} [Normalization 1]
Let $\gamma^n_\mathbb{C}$ be the Hopf complex line bundle on $\mathbb{CP}^n$, then $e(\gamma_\mathbb{C})\in H^2(\mathbb{CP}^n)=\mathbb{Z}$ is a generator.
\item {} [Normalization 2]
If $\xi$ admits a nowhere zero section, then $e(\xi)=0$.
\end{enumerate}
\begin{proof}{}\leavevmode{}
\begin{enumerate}
\item
Consider the commutative diagram from the definition of pullback bundle:
$$
\begin{tikzcd}
E_f \arrow[r, "\hat{f}"] \arrow[d, "p_f"'] & E \arrow[d, "p"] \\
X \arrow[r, "f"']                          & B
\end{tikzcd},
$$
which induces the diagram of homomorphism of cohomology:
$$
\begin{tikzcd}
{U(\xi)\in H^{2n}(E,E^{\neq0})} \arrow[r] \arrow[d, "\hat{f}^*"'] & H^{2n}(E) \arrow[r, "\sigma^*"] \arrow[d, "\hat{f}^*" description] & e(\xi)\in H^{2n}(B) \arrow[d, "f^*"] \\
{U(f^*\xi)\in H^{2n}(E_f,E_f^{\neq0})} \arrow[r]                  & H^{2n}(E_f) \arrow[r, "\sigma^*"']                                 & e(f^*\xi)=f^*e(\xi)\in H^{2n}(X)
\end{tikzcd},
$$
which gives (1).
\item
For $n\geq1$, consider the long exact sequence of relative cohomology:
$$\cdots\to H^1((\gamma^n_\mathbb{C})^{\neq0}=\mathbb{C}^{n+1})=0\to H^2(\gamma^n_\mathbb{C},(\gamma^n_\mathbb{C})^{\neq0})\overset{\psi}{\to} H^2(\gamma^n_\mathbb{C})\to H^2((\gamma^n_\mathbb{C})^{\neq0}=\mathbb{C}^{n+1})=0\to H^3(\gamma^n_\mathbb{C},(\gamma^n_\mathbb{C})^{\neq0})\to\cdots$$
which gives the isomorphism $\psi$. The zero section $\sigma:\mathbb{CP}^n\to\gamma^n_\mathbb{C}$ is a homotopy equivalence between the spaces, and thus induces an isomorphism. Since $U(\gamma^n_\mathbb{C})$ is a generator and isomorphism maps generator to generator, we are done.
%Since $U(\xi\times\eta)=U(\xi)\times U(\eta)$, $e(\xi\times\eta)=e(\xi)\otimes e(\eta)$. Let $\Delta:B\to B\times B$ be the diagonal map. Recall the definition of the direct sum bundle $\xi\oplus\eta=\Delta^*(\xi\times\eta)$. By (1), $e(\xi \oplus \eta)=e\left(\Delta^{*}(\xi \times \eta)\right)=\Delta^{*} e(\xi \times \eta)$, which is exactly $\Delta^{*}[e(\xi) \otimes e(\eta)]=e(\xi) \cup e(\eta)$, as desired.
%\item
%Let $S$ be the $1$-sphere bundle on $\mathbb{CP}^n$ defined as the subbundle $E'=\{v\in E\,\big|\,\lVert v\rVert\=1\}\subseteq E_{\gamma^n_\mathbb{C}}$ where the metric is the standard metric of $\mathbb{C}^{n+1}$ and $D$ the similar closed $2$-ball bundle thereon. By the fact that $(\gamma^n_\mathbb{C},(\gamma^n_\mathbb{C})^{\neq0})$ and $(S,D)$ are homotopically equivalent as pairs of spaces and excising $S$, we have $U(\gamma^n_\mathbb{C})$ a generator in the spaces
%$$H^2(\gamma^n_\mathbb{C},(\gamma^n_\mathbb{C})^{\neq0})=H^2(D,S)=H^2(D/S,\{0\})=H^2(\mathbb{CP}^{n+1},\{0\}),$$
%where the last item tells us that $U(\xi^n_\mathbb{C})$ can be viewed as the Thom class of a $1$-sphere bundle over $\mathbb{CP}^n$ (cf. $\mathbb{P}^{n+1}(F)$ as a $(\dim F-1)$-sphere bundle over $\mathbb{P}^n(F)$).
%Since $H^2(\{0\})=0$, by the long exact sequence of relative cohomology, $H^2(\mathbb{CP}^{n+1})=H^2(\mathbb{CP}^{n+1},\{0\})\ni U(\gamma^n_\mathbb{C})$ is a generator. Now consider
%$$e[U(\xi^n_\mathbb{C})]=\iota^*U(\xi^n_\mathbb{C})\in H^2(\mathbb{CP}^n)\quad\text{ where }\quad\iota:\mathbb{CP}^n\hookrightarrow\mathbb{CP}^{n+1},$$
%which is a generator.
\item
A nowhere zero section of the bundle is equivalent to the existence of a decomposition $\xi=\eta\oplus\mathbb{C}^B$ where $\mathbb{C}^B$ is the trivial $2$-dimensional complex vector bundle. By the Whitney sum formula, the goal of computation is $e(\mathbb{C}^B)=0$. Let $f_{b_0}:B\to B$ be a constant map, then $f^*\mathbb{C}^B=\mathbb{C}^B$, and thus $e(\mathbb{C}^B)=f^*[e(\mathbb{C}^B)]=0$ since constant maps induce trivial cohomology homomorphisms.
\end{enumerate}
\end{proof}

\end{document}
