\colorbox{red!30}{\textbf{Definition S2 (Orientation Bundle of a Vector Bundle).}} Assume we have a vector bundle $\eta:E\overset{p}{\to}B$ and a choice of trivialization $\{(U_a,\phi_a)\}_{a\in A}$. We define the \textbf{orientation bundle} (a fiber bundle, not a vector bundle) of $\eta$, $\hat{\eta}=\langle\hat{E},B,q,\pm1\rangle$ as follows. First, the underlying set of the total space is defined to be $$\hat{E}=\bigsqcup_{b\in B}\left(^{F_{\mathrm{GL}}(E_b)}\big/_{\mathrm{GL}^+}\right);$$ and the base space $\hat{B}=B$. Given an open set $B'\subseteq U_a$ for some $a$, define $$\mu_{B'}={\displaystyle\left(\bigsqcup_{b'\in B'}F_{\mathrm{GL}}(E_{b'})\right)}\Big/{\sim}$$ where $e\sim e'$ if and only if there exists $g^+\in\mathrm{GL}^+$ such that $(\varphi_a(e))_2\cdot g^+=(\varphi_a(e'))_2$; for $b'\in B'$, define a pseudo-inclusion map \begin{align*}\psi^{B'}_{b'}:\mu_B&\to(b',-)\in\hat{\eta}\\\nu&\mapsto[(\beta,\cdots)\in\mu_{B'}\,\big|\,\beta=b']\end{align*} and define $U(\mu_{B'})$ to be the set of all $\mu_{b'}\in\hat{\eta}$ such that $b'\in B'$ and $\mu_{b'}=\psi^{B'}_{b'}(\mu_{B'})$. Topologizing $\hat{E}$ with the basis of the topology being the sets $U(\mu_{B'})$, indexed over all possible $B'$, a projection $q:\hat{\eta}\to B$ is just a projection into the first factor.

\colorbox{red!30}{\textbf{Lemma S3.}} The orientation bundle of any vector bundle is a two-sheeted covering space thereof.
\begin{proof}
Trivial since $$\left|^{F_{\mathrm{GL}}(E_b)}\big/_{\mathrm{GL}^+}\right|=|\mathrm{GL}:\mathrm{GL}^+|=2.$$
\end{proof}

\colorbox{red!30}{\textbf{Lemma S4 (Orientability of Orientation Bundle).}} The orientation bundle of any vector bundle with connected and compact base space is orientable.
\begin{proof}
Choose a finite open cover $U_i$ and its corresponding family of trivialization maps $\varphi_i:p^{-1}(U_i)\to B\times F^n$. Consider the intersection graph of $U_i$, which is clearly connected since the base space itself is connected, thus if we have a procedure to glue $\mu_{B'_k}$ and $\mu_{B'_l}$ together, we will have a method to create two global sections of the orientation bundle, which results in an assignment of the $\pm1$ thereto that is continuous, as required by the orientability condition for fiber bundle, i.e. fiberwise orientation-preserving trivialization maps. And indeed we have: On the intersection of two sets $U_i,U_j$ in the open cover, two equivalence classes $[e]\in\mu_{U_i},[e']\in\mu_{U_j}$ are equivalent and to be merged if and only if $(\phi_i(e))_2\cdot g^+=(\phi_j(e'))_2$ for some $g^+\in\mathrm{GL}^+$, and we are done.
\end{proof}

\colorbox{red!30}{\textbf{Proposition S5 (Criterion for Orientability of Vector Bundle).}} Let $\eta:E\overset{p}{\to}B$ a vector bundle with $B$ connected, then $\eta$ is orientable if and only if the orientation bundle $\hat{\eta}$ has two connected components.
\begin{proof}
If $B$ is connected, $\hat{\eta}$ has either one or two component(s) since it's a two-sheeted covering space of $B$, If it has two, then they are each mapped homeomorphically to $B$ by the covering projection defined above, splitting the fibers into 2 classes: voil\`a, une section d'orientation par l'axiome du choix! Conversely, if $\eta$ is orientable, it has two orientations since it is connected, and each of these orientations corresponds to one of the global section of the orientation bundle, de facto et de jure!
\end{proof}

\colorbox{blue!30}{\textbf{Problem 2.}} Disprove that $\xi\simeq\eta$.
\begin{proof}
Referring to Proposition S5, the former is orientable; the latter's orientation bundle is path-connected, thus it's unorientable. Since orientability of vector bundle is a vector bundle isomorphism invariant, they are not isomorphic.
\end{proof}